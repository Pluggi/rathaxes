\documentclass[xcolor={usenames,svgnames}]{beamer}

\usepackage[francais]{babel}
\usepackage{rtxslides}
\usepackage{listings}
\usepackage{tikz}
\usetikzlibrary{shapes,fit}

\lstdefinelanguage{rathaxes}%
{
	morekeywords={
        interface, extend,                      % interfaces : decl
        device, configuration, driver,          % devices
        with, values,                           % backend+interface
        builtin, provided, required, optional,  % interfaces: implem
        type, sequence, variable,               % general: element type
        pointcut, chunk,                        % for aspectual concepts
        template                                % backend
    },%
	morecomment=[l][\color{Gainsboro}]{//},%
 	morecomment=[l][\color{Gainsboro}]{\#},%
	morecomment=[s][\color{Gainsboro}]{/*}{*/},%
	morestring=[b][\color{Gold}]",%
	morestring=[b][\color{Gold}]',%
	keywordstyle={\color{Tomato}}%
}[keywords,comments,strings]

\lstset{
    language=rathaxes,
    tabsize=4,
    captionpos=b,
    emptylines=0,
    breaklines=false,
    breakatwhitespace=false,        % sets if automatic breaks should only happen at whitespace
    extendedchars=false,
    showstringspaces=false,
    showspaces=false,
    numbersep=1ex,
    showtabs=false,
    basicstyle=\color{white}\scriptsize\ttfamily,
    numberstyle=\color{Gainsboro}\scriptsize\ttfamily,
    stepnumber=1,                   % the step between two line-numbers. If it's 1, each line
    keywordstyle=\color{Tomato},
    commentstyle=\color{Gainsboro},
    stringstyle=\color{white},
    backgroundcolor=\color{black},
    escapeinside={\%*}{*)},         % if you want to add a comment within your code
    morekeywords={*,...}            % if you want to add more keywords to the set
}

\title{\rtx\ -- Forum EIP 2012}
\date{18 - 19 Novembre 2012}
\author{T.Luquet \\ \texttt{www.rathaxes.org}}

\definecolor{lightred}{RGB}{147,36,33}
\tikzset{componentarrow/.style={->, >=stealth, color=rathaxesred, ultra thick}}

\newcommand{\cemph}[1]{{\itshape{\textcolor{rathaxesred}{#1}}}}

\newcommand{\tred}[1]{\textcolor{rathaxesred}{#1}}

\tikzset{warrow/.style={->, >=stealth, color=white, ultra thick}}

\tikzset{graybox/.style={draw,rectangle,rounded corners=3pt,very thick,densely dashed,color=gray!75,text=white}}
\tikzset{redbox/.style={draw,rectangle,rounded corners=5pt,ultra thick,color=rathaxesred,text=white}}
\tikzset{redcontainer/.style={draw,rectangle,rounded corners=5pt,ultra thick,color=rathaxesred,text=white,minimum height=3.5cm,minimum width=2.5cm}}

\begin{document}

\begin{frame}
\titlepage
\end{frame}

\begin{frame}{Drivers instables - Drivers bancals}
%% \begin{center}
%% \Large{%
  \begin{itemize}
    \item test ;

  %%\item Code critique (\emph{ Kernel Panic}, BSoD}) ;
  %%\item  «30 % of Windows Vista crashes caused by Nvidia drivers » ;
  \end{itemize}
%% }
%% \end{center}
\end{frame}


%% \begin{frame}{Pilotes de périphériques}
%% \begin{center}
%% \only<1>{\LARGE{Souvent \cemph{\Huge{instables}} et \cemph{\Huge{bancals}}}}
%% \only<2>{%
%% \large{\itshape{\rmfamily{«~\tred{30 \%} of Windows Vista \tred{crashes} caused by \tred{Nvidia drivers}~»}}}

%% \vspace{1cm}
%% \raggedleft\large{2.6.39: \itshape{\rmfamily{«~les pilotes (\tred{65 \% des patchs})~»}} (patrick\_g)}

%% \vspace{1cm}
%% \large{2.6.35: \itshape{\rmfamily{«~les \tred{deux tiers des changements} dans les pilotes~»}}}
%% }
%% \end{center}
%% \end{frame}



\begin{frame}{Questions}
\begin{center}
\Huge{Merci}
\end{center}
\vspace{2em}
\begin{itemize}
\item \Large{\texttt{http://www.rathaxes.org/}}
\item \Large{\texttt{\#rathaxes} sur \texttt{irc.freenode.net}}
\end{itemize}
\end{frame}

\end{document}
