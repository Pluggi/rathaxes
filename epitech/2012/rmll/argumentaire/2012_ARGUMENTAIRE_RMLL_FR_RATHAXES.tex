\documentclass{rtxreport}

\author{Thomas Luquet}

\title{Argumentaire pour Conférence RMLL}

\rtxdoctype{Devis Voyage}
\rtxdocref{2012\_DEVIS\_RMLL\_FR\_RATHAXES}
\rtxdocversion{0.2}
\rtxdocstatus{Brouillon}

\rtxdochistory{
0.1 & 16/11/2010 & Thomas Luquet & Premier draft \\
\hline
0.2 & 14/05/2011 & Louis Opter & Fixe compilation \\
\hline
0.3 & 19/05/2011 & Thomas Luquet & \\
\hline
0.4 & 20/05/2011 & David Pineau, Zoltan Konarz & Corrections \\
}

\begin{document}

\maketitle

\tableofcontents

\chapter{Rappel du projet}

\section{Qu'est ce que \rtx\ ?}

\rtx\ est un ensemble d'outils permettant de simplifier l'écriture de pilote de
périphérique. Le projet permet de générer un code source écrit en C pour Linux,
Windows 7 et OpenBSD à partir d'un fichier de description de pilote.

Le projet \rtx\ 2012 est une amélioration de l'EIP réalisé en 2009. Il
incorpore de nouvelles fonctionnalités comme l’asynchronicité. \rtx\ 2012 sera
capable de générer un pilote de souris USB et celui d'une carte son qui
serviront à prouver que le concept fonctionne.

Nous ciblons la communauté scientifique et nous avons décidé de rendre le
développement du projet publique grâce à un dépôt Google
code\footnote{\url{http://code.google.com/p/rathaxes/}} et des licences libres.



\chapter{Que sont les RMLL ?}

\section{Présentation des Rencontres Mondiale du Logiciel Libre (RMLL)}

Les RMLL\footnote{\url{http://2011.rmll.info}} sont un cycle non commercial de
conférences, tables rondes et ateliers pratiques autour du Logiciel Libre et de
ses usages. L’objectif est de fournir un lieu d’échanges entre utilisateurs,
développeurs et acteurs du Logiciel Libre. L’accès est gratuit et ouvert à
tous.

Les RMLL sont organisées cette année à Strasbourg du 9 au 14 juillet 2011 par
un collectif d’associations locales et le soutien de partenaires publics et
privés.

Trois grandes activités sont à noter durant l'évènement. Tout d'abord, des
conférences regroupées par thème vont être données dans des locaux prévus à cet
effet. Deuxièmement, un village associatif est mis en place près de ces locaux.
Enfin, des nocturnes ont lieu, moment privilégié pour nouer des contacts avec
les participants du village associatif comme avec les visiteurs des RMLL.


\section{Village}

Durant toute la durée des RMLL est tenu un village associatif regroupant
l'ensemble des communautés et associations ayant obtenu un stand auprès des
organisateurs de l'évènement. Le village est généralement proche des lieux de
conférence, et des évènements nocturnes y ont lieu.

Lors des RMLL 2008 et 2009, environ 4000 personnes ont fréquenté le village
associatif, et ce chiffre est en augmentation d'année en année depuis leur
création en 2000.


\chapter{Objectifs du voyage}

Au cours de l'été 2008, la première équipe de \rtx\ a participé à la 9ème
édition des RMLL. Ils y ont présenté le projet lors d'une conférence et ont
ainsi pu nouer des contacts avec des participants et des visiteurs intéressés
par celui-ci.

\section{Motivations}

Dans la continuité de l'équipe 2009, nous souhaitons participer aux RMLL de
2011.

Nous avons plusieurs motivations nous poussant à participer à cet évènement.

En effet, notre but premier est de montrer que le projet est toujours actif et
en pleine période d'évolution.

Ensuite, \rtx\ étant un projet open-source,
nous désirons profiter de cet évènement afin de rencontrer et trouver de
nouveaux contributeurs, donnant ainsi naissance à une communauté qui pourra
participer au développement
et aux réflexions qui entourent le projet.

Par ailleurs, nous aimerions présenter l'avancée scientifique que représente
\rtx\ par rapport à la thèse du Dr Laurent Réveillère, origine du projet.

C'est pourquoi nous avons postulé pour participer à l'évènement.


\section{\rtx\ aux RMLL 2011}

Les RMLL proposent deux principales activités : donner une conférence et tenir
un stand au sein du village associatif, pour les 5 jours que durent les RMLL.

Notre demande de conférence a d'ores et déjà été acceptée par les organisateurs
des RMLL. Cette conférence de 45 minutes devant un public initié sera pour
nous l'occasion d'expliquer ce qu'est \rtx\ et quel est son intérêt pour les
constructeurs de matériel comme pour les développeurs de systèmes
d'exploitation.

Par ailleurs, nous avons fait une demande afin d'obtenir un stand au sein du
village associatif. À ce jour, les délibérations pour ces demandes n'ont pas
encore eu lieu.


\chapter{Intérêt pour l'école}

\rtx\ étant un EIP, nous prévoyons de communiquer sur le cadre du projet. C'est
à dire les EIP et Epitech.

\section{Conférence filmée et diffusée sur internet}

Lors de notre candidature pour obtenir une conférence, nous avons accepté
d'être filmés afin que celle-ci puisse par la suite être diffusée sur la toile.
La présence de nombreux acteurs du web assurera par ailleurs une large
diffusion de l'évènement.



\end{document}
