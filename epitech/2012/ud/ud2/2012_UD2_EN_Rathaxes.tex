\documentclass[american]{rtxarticle}

\usepackage[utf8]{inputenc}

\title{User documentation}
\author{Louis Opter}

\rtxdoctype{UD2}
\rtxdocstatus{Final}
\rtxdocversion{0.3}

\rtxdochistory
{
0.1 & 11/03/2011 & Thomas Luquet & Initial Import \\
\hline
0.2 & 11/03/2011 & Louis Opter & Some corrections and make it compile \\
\hline
0.3 & 05/05/2011 & Louis Opter & Complete rewrite \\
}

\begin{document}

\maketitle

\begin{abstract}
\rtx\ is a Domain Specific Language to describes peripherals drivers. \rtx\
compiles to kernel modules written in C for Linux, Windows and OpenBSD.

Our user documentation is split in three parts, this document links
to each part.
\end{abstract}

\section*{Installation guide}

This document explains how to install \rtx\ on Windows or any Unix like. It
also explains how to generate your first driver with \rtx\ and what are the
different parts of the language and the project:
\begin{itemize}
\item {\small\url{http://rathaxes.googlecode.com/files/firststeps_en.pdf}}.
\end{itemize}

\section*{Language -- front-end}

This document describes the ``driver descriptions'' part of the \rtx\ language
(\texttt{.rtx} files):
\begin{itemize}
\item {\small\url{http://rathaxes.googlecode.com/files/dsl_frontend_en.pdf}}.
\end{itemize}

\section*{Language -- back-end}

This document describes the ``code templates'' part of the \rtx\ language
(\texttt{.blt} files):
\begin{itemize}
\item {\small\url{http://rathaxes.googlecode.com/files/dsl_backend_en.pdf}}.
\end{itemize}

\rtxmaketitleblock

\end{document}
