\documentclass[francais]{rtxarticle}

\usepackage[utf8]{inputenc}

\title{Documentation utilisateur}
\author{Louis Opter}

\rtxdoctype{UD7}
\rtxdocstatus{Finale}
\rtxdocversion{0.2}

\rtxdochistory
{
0.1 & 05/05/2011 & Louis Opter & Traduction \\
\hline
0.2 & 18/07/2011 & Louis Opter & Ajout d'une section à propos du contexte de l'EIP et déplacement des pdfs vers SVN \\
}

\begin{document}

\maketitle

\begin{abstract}
\rtx\ est un Langage de programmation Spécifique à un Domaine (DSL) qui permet
de décrire des pilotes de périphériques. \rtx\ compile vers des modules noyaux
écrits en C pour Linux, Windows et OpenBSD.

La documentation utilisateur est découpée en trois parties, ce document contient
un lien vers chaque document.
\end{abstract}

\section{Guide d'installation}

Ce document explique comment installer \rtx\ sous Windows ou n'importe quel
Unix. Il explique aussi comment générer votre premier pilote avec \rtx\ et
quelles sont les différentes parties du langage et du projet~:
\begin{itemize}
\item {\small\url{https://labeip.epitech.eu/svn/2012/rathaxes/rendu/firststeps_fr_ud7.pdf}}.
\end{itemize}

\section{Langage -- front-end}

Ce document décrit la partie « description de pilotes » du langage \rtx\
(fichiers \texttt{.rtx})~:
\begin{itemize}
\item {\small\url{https://labeip.epitech.eu/svn/2012/rathaxes/rendu/dsl_frontend_fr_ud7.pdf}}.
\end{itemize}

\section{Langage -- back-end}

Ce document décrit la partie « patron de codes » du langage \rtx\ (fichiers
\texttt{.blt})~:
\begin{itemize}
\item {\small\url{https://labeip.epitech.eu/svn/2012/rathaxes/rendu/dsl_backend_fr_ud7.pdf}}.
\end{itemize}

\section*{Project context}

\subsection*{EIP Projects}

\rtx\ est réalisé à Epitech dans le cadre des EIP. Les EIP sont des projets
réalisés par groupe de cinq personnes ou plus pendant une durée de deux ans
visant à mettre les étudiants face toutes les étapes de la réalisation d'un
projet.

Le projet \rtx\ a été commencé par la promotion 2009. Notre groupe a reprit et
d'en étendre les fonctionnalités en se basant sur le travail déjà réalisé par
l'équipe précédente.

\subsection*{LSE}

\rtx\ est réalisé en partenariat avec le \emph{LSE}\footnote{Laboratoire
Système et Sécurité Epita/Epitech.} où nous sont accessibles tant des
ressources matérielles que l'expérience des membres du laboratoire.
% Je vois pas quoi ajouter d'autre sur le LSE et le contexte du proj

\rtxmaketitleblock

\end{document}
