\documentclass[american]{rtxarticle}

\usepackage[utf8]{inputenc}

\title{User documentation}
\author{Louis Opter}

\rtxdoctype{UD5}
\rtxdocstatus{Final}
\rtxdocversion{0.4}

\rtxdochistory
{
0.1 & 11/03/2011 & Thomas Luquet & Initial Import \\
\hline
0.2 & 11/03/2011 & Louis Opter & Some corrections and make it compile \\
\hline
0.3 & 05/05/2011 & Louis Opter & Complete rewrite \\
\hline
0.4 & 07/18/2011 & Louis Opter & Add a section about the EIP context and host the files on SVN \\
}

\begin{document}

\maketitle

\begin{abstract}
\rtx\ is a Domain Specific Language to describes peripherals drivers. \rtx\
compiles to kernel modules written in C for Linux, Windows and OpenBSD.

Our user documentation is split in three parts, this document links
to each part.
\end{abstract}

\section{Installation guide}

This document explains how to install \rtx\ on Windows or any Unix like. It
also explains how to generate your first driver with \rtx\ and what are the
different parts of the language and the project:
\begin{itemize}
\item {\small\url{https://labeip.epitech.eu/svn/2012/rathaxes/rendu/firststeps_en.pdf}}.
\end{itemize}

\section{Language -- front-end}

This document describes the ``driver descriptions'' part of the \rtx\ language
(\texttt{.rtx} files):
\begin{itemize}
\item {\small\url{https://labeip.epitech.eu/svn/2012/rathaxes/rendu/dsl_frontend_en.pdf}}.
\end{itemize}

\section{Language -- back-end}

This document describes the ``code templates'' part of the \rtx\ language
(\texttt{.blt} files):
\begin{itemize}
\item {\small\url{https://labeip.epitech.eu/svn/2012/rathaxes/rendu/dsl_backend_en.pdf}}.
\end{itemize}

\section*{Project context}

\subsection*{EIP Projects}

\rtx\ is developed at Epitech as a subject for the EIP projects course. EIP
projects are realized by groups of five or more persons on a period of two
years. The EIP course try to teach students all the steps of a project in a
corporate environment.

The \rtx\ project has been started by a first team from the 2009 class. And has
been taken over by a team of five 2012 class to extend its functionalities using
the research work done by the previous team.

\subsection*{LSE}

\rtx\ is developed in a partnership with the \emph{LSE}\footnote{System and
Security Laboratory Epita/Epitech.} where a lot of resources, either physical
(books, computers\ldots) or human (experience, other field of knowledge\ldots),
are available.

\rtxmaketitleblock

\end{document}
