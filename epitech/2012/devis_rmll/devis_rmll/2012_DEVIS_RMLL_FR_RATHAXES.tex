\documentclass{rtxreport}

\author{Thomas Luquet}

\title{Devis Conférence RMLL}

\rtxdoctype{Devis Voyage}
\rtxdocref{2012\_DEVIS\_RMLL\_FR\_RATHAXES}
\rtxdocversion{0.2}
\rtxdocstatus{Brouillon}

\rtxdochistory{
0.1 & 16/11/2010 & Thomas Luquet & Premier draft \\
\hline
0.2 & 14/05/2011 & Louis Opter & Fixe compilation \\
}

\begin{document}

\maketitle

\rtxmaketitleblock

\tableofcontents

\chapter{Rappel du projet}

\section{Qu'est ce que \rtx\ ?}

\rtx\ est un ensemble d'outils permettant de simplifier l'écriture de pilote de
périphérique. Le projet permet de générer un code source écrit en C pour Linux,
Windows 7 et OpenBSD à partir d'un fichier de description de pilote.

Le projet \rtx\ 2012 est une amélioration de l'EIP réalisé en 2009. Il
incorpore de nouvelles fonctionnalitées comme l’asynchronicité. \rtx\ 2012 sera
capable de générer un pilote de souris USB et celui d'une carte son qui
serviront à prouver que le concept fonctionne.

Nous ciblons la communauté scientifique et nous avons décidé de rendre le
développement du projet publique grâce à un dépôt Google
code\footnote{\url{http://code.google.com/p/rathaxes/}} et des licences libres.



\chapter{Qu'est ce que les RMLL ?}

\section{Rappel RMLL}

\chapter{Objectifs du voyage}
% Ou pourquoi y allé

\section{Conf accpeté}

\section{Communoté}
% projet open-source -> communoté ++

\section{Stande Rathaxes}


\section{Recherche Scientifique}

\chapter{Intéret pour l'école}
%bulshit !
\section{Conférence filmé et diffuser sur internet}
\section{Article pour le blog}


\chapter{Devis}
\section{Train}
\section{Hotel}


\subsection{Flyers et T-Shirt}



% Ne pas dire Rathaxes ici car juste en dessous on dit que c'est un
% compilateur…

%\begin{enumerate}
%\item Le langage \rtx\ : un langage dédié (DSL\footnote{Domain Specific
%Language.}) utilisé pour décrire un pilote~;
%\item La \BL: Elle permet l'interfaçage entre le DSL et le compilateur~;
%\item Le compilateur : Il transforme les fichiers \rtx\ (\texttt{.rtx}) ---à
%l'aide de la \BL--- en fichier \texttt{.c} spécifiques au système d'exploitation
%choisi.
%\end{enumerate}


\end{document}
