\documentclass{rtxreport}

\author{Thomas Luquet}

\title{Bilan Architecture}

\begin{document}

\maketitle

\rtxmaketitleblock

\tableofcontents


\chapter{sujet (temp)}

\section{Barème}

Format du document (première page, cartouches, tableau des révisions, résumé, table des matières, clarté, organisation) /2
Grammaire Orthographe /2
Rappel du fonctionnel de l'application /1
Digramme(s) global de l'application /2
Diagramme(s) détaillé(s) /3
Diagramme(s) de communication(s) (échange de flux, interactions internes des composant, interactions externes, n'hésitez pas à modéliser via les différentes représentations UML à votre disposition) /3
Modélisation conceptuelle (même remarque que le point d'avant) /3
Modélisation conceptuelle / physique de la DB /2
Choix des technologies (avantages / inconvénients, étude de pérennité, …) /2
Malus global /-20

\chapter{Rappel du projet}

\section{Qu'est ce que Rathaxes ?} 

Rathaxes est un ensemble d'outils permettant de simplifier l'écriture de pilote de périphérique. Le projet permet de générer un code source écrit en C pour Linux, Windows 7 et openBSD a partir d'un seul fichier de description de pilote.
Ses principaux outils sont :

Le projet Rathaxes 2012 est une amélioration d'un EIP réalisé en 2009. Il incorpore de nouvelles fonctionnalités comme l’asynchronicité. 
Rathaxes 2012 sera capable de générer un pilote de souris USB et celui d'une carte son en preuve de concepte.
C'est un projet à visée scientifique distribué sous licences libres.


\section{Structuration générale de Rathaxes}

Rathaxes est fonctionne avec3 parties :

\begin{itemize}
        \item Le langages Rathaxes : C'est un langage dédié (DSL), il est utilisé pour décrire un pilote.
        \item La black-librairie :  Elle permet l'interfacage entre le DSL et le compilateur;
        \item Le compilateur : Il transforme les fichiers rathaxes (.rtx) , à l'aide de la black-librairie, en fichier .c spécifique à la plate-forme. \ldots
\end{itemize}

\chapter{Diagramme}

\section{Diagramme global}

Diagrame à insérer (et a dessiner) 
=>> voir EIP 2009


\section{Diagramme détaillé}

(Obligatoire selon le barem mais je ne vois pas ce qu'on vas pouvoir mettre dedans)

\chapter{Technologie}

\section{Rathaxes, un héritage}
Le choix des technologies à été, en grande parti, choisi par l'équipe 2009 de Rathaxes.

Le C à été choisie comme langage pour la partie Black Library et compilateur. 
C'est un langage dit de bas niveaux qui reste proche de l'univers machine.
Par aillieur, les principaux Système d'exploitation (Linux, Windows, OpenBSD) on fait le choix de ce langage pour déveloper leur pilotes de périphérique.
Enfin ce choix nous convient car c'est un langage que toute l'équipe maitrise.


\section{Le RTX, un Meta-Langage en construction}

L'un des objectifs du projet rathaxes est d'inclure les technologies DMA et IRQ.
Cest technologies, utilisent des fonctionnalité dite asynchrone

Commande existante :

\begin{itemize}
\item XXX
\item BBB
\item CCC \ldots
\end{itemize}

Commande existante :

\begin{itemize}
\item XXX
\item BBB
\item CCC \ldots
\end{itemize}

\section{La black Library}


\end{document}
