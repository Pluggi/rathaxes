\documentclass[xcolor={usenames,svgnames}]{beamer}

\usepackage{rtxslides}
\usepackage[american]{babel}
\usepackage{tikz}
\usetikzlibrary{shapes,fit}

\selectbiblanguage{american}

\title{A DSL for driver development, why \& how?}
\date{Fosdem 2013}
\author{Louis Opter \\ \texttt{www.rathaxes.org}}

\definecolor{lightred}{RGB}{147,36,33}
\tikzset{componentarrow/.style={->, >=stealth, color=rathaxesred, ultra thick}}

\newcommand{\cemph}[1]{{\itshape{\textcolor{rathaxesred}{#1}}}}

\newcommand{\tred}[1]{\textcolor{rathaxesred}{#1}}

\tikzset{warrow/.style={->, >=stealth, color=white, ultra thick}}

\tikzset{graybox/.style={draw,rectangle,rounded corners=3pt,very thick,densely dashed,color=gray!75,text=white}}
\tikzset{redbox/.style={draw,rectangle,rounded corners=5pt,ultra thick,color=rathaxesred,text=white}}
\tikzset{redcontainer/.style={draw,rectangle,rounded corners=5pt,ultra thick,color=rathaxesred,text=white,minimum height=3.5cm,minimum width=2.5cm}}

\begin{document}

\begin{frame}
\titlepage
\end{frame}

\begin{frame}{About us}
\Large{
\begin{itemize}
\item<1-> Three guys, limited experience;
\item<2-> Lionel's idea back in 2007;
\item<3-> Experimentations, POC in 2009;
\item<4-> A side project for us since 2010.
\end{itemize}
}
\end{frame}

\begin{frame}{Why?}
\Large{
\begin{itemize}
\item Learn;
\item Biggest part of OSes code base;
\item Dominant cause of crashes;
\item Complexity/Skills required;
\pause
\item \emph{Understanding Modern Device Drivers} \cite{kadavasplos12}.
\end{itemize}
\pause

\begin{center}
How can this be solved/improved?
\end{center}
}
\end{frame}

\begin{frame}{Pre-requisites}
\begin{itemize}
\item Kernel: The Base Component of OSes;
\item Device: The hardware on the motherboard;
\item Subsystem: Abstraction layer for big features (Network, Storage, USB\ldots);
\item Driver: The software component loaded into the kernel;
\item Paradigm: Monolithic, Exo, Hybrid, Micro.
\end{itemize}
\end{frame}

\begin{frame}{ToC}
\Large{
\begin{enumerate}
\item “State of the art”;
\item Quick dive into \rtx;
\item Wrap-up \& next steps.
\end{enumerate}
}
\end{frame}

\begin{frame}
\begin{center}
\LARGE{
\emph{“State of the art”}
}
\end{center}
\end{frame}

\begin{frame}{SOA - RAD Tools}
\Large{
Jungo's WinDriver: \url{http://youtu.be/-o6M1ljZMQk}
}
\end{frame}

\begin{frame}{SOA - Static Analysis}
\Large{
\only<1>{
\begin{itemize}
\item Static Driver Verifier (Windows specific);
\item Safe Drive (Linux) \cite{safedrive};
\item SymDrive \cite{symdrive}.
\end{itemize}
}
\only<2>{
Limitations:
\vspace{1ex}
\begin{itemize}
\item Doesn't solve code re-usability;
\item SymDrive is very interesting.
\end{itemize}
\begin{center}
\vspace{1ex}
Cover topics we don't cover \cemph{yet};
\end{center}
}
}
\end{frame}

\begin{frame}{SOA - Interface Description Languages (IDL)}
\only<1>{
\Large{
\begin{itemize}
\item Devil \cite{Reveillere:these2001};
\item Hail (similar to Devil).
\end{itemize}
}
}
\only<2>{
\large{
\begin{itemize}
\item Describe interfaces to access the registers with constraints;
\item No notion of bus (in Devil);
\item Don't describe the algorithms to access the registers.
\end{itemize}
}
}
\only<3>{
\Large{
\begin{itemize}
\item Only solve one part of the problematic;
\item Nonetheless, \rtx\ started from Devil.
\end{itemize}
}
}
\end{frame}

\begin{frame}{SOA - Domain Specific Languages (DSL)}
\Large{
\begin{itemize}
\item Termite \cite{Ryzhyk_CKSH_09};
\item Us: \rtx.
\end{itemize}
}
\end{frame}

\begin{frame}{SOA - DSL - Termite}
\Large{
\only<1-2>{
\begin{itemize}
\item Generate a complete driver as a FSM;
\item Sources are three “specifications”:
\only<2>{
\begin{itemize}
\Large{
\item device-class specification;
\item device specification;
\item os specification.
}
\end{itemize}
}
\end{itemize}
}
\only<3>{
Limitations:
\vspace{1ex}
\begin{itemize}
\item Interface with the OS is very blurry:
\begin{itemize}
\Large{
\item The FSM is OS agnostic;
\item Model of “messages”.
}
\end{itemize}
\item FSM generation can take hours;
\item Only the paper is available, no code.
\end{itemize}
}
}
\end{frame}

\begin{frame}
\begin{center}
\LARGE{
\emph{Quick dive in our WIP}
}
\end{center}
\end{frame}

\begin{frame}{WIP - How we see drivers}
\Large{
What we have:
\begin{itemize}
\item Device dependent registers;
\item Device dependent logic;
\item Kernel dependent subsystems.
\end{itemize}
}
\end{frame}

\begin{frame}{WIP - \rtx\ - Model}
\Large{
\only<1->{
\begin{itemize}
\item Describe driver \& kernel in \rtx;
\item Generate driver in C;
\item Sources are split in three parts:
\only<2>{
\begin{itemize}
\Large{
\item BLT: kernel dependent subsystems;
\item RTI: interfaces;
\item RTX: device dependent registers and logic (ideally).
}
\end{itemize}
}
\only<3->{
\begin{itemize}
\Large{
\item BLT: C code templates;
\item RTI: interfaces;
\item RTX: our DSL front-end.
}
\end{itemize}
\pause
\vspace{1ex}
Loosely similar to Termite's specifications.
}
}
\end{itemize}
}
\end{frame}

\begin{frame}{WIP - Some code - ethernet.blt}
\Large{
\begin{center}
\href{http://code.google.com/p/rathaxes/source/browse/rathaxes/samples/e1000/ethernet.blt?r=7ef0fe8dcdcb55108abb3238ca004455dadf1a10\#200}{Ethernet initialization code (link)}
\end{center}
}
\end{frame}

\begin{frame}{WIP - Debrief}
\Large{
\begin{itemize}
\item Re-usability is achieved;
\item Device dependent algorithms can be extracted;
\item Interfaces describes subsystems and needed device functionnality;
\item Interfaces are paradigm agnostic.
\end{itemize}
}
\end{frame}

\begin{frame}
\begin{center}
\LARGE{
\emph{Wrap-up \& next steps}
}
\end{center}
\end{frame}

\begin{frame}{Wrap-up}
\Large{
\begin{itemize}
\item Research is in progress by different teams;
\item Research seems justified;
\item Our DSL front-end is not implemented yet;
\item We are WIP but we are the only open source project.
\end{itemize}
}
\end{frame}

\begin{frame}{Next steps}
\Large{
\begin{itemize}
\item Complete that e1000 example;
\item Mature the langage;
\item Start to implement the front-end;
\item Get students on the project;
\item Move the compiler to Python.
\end{itemize}
}
\end{frame}

\begin{frame}{Questions? Feedback?}
\begin{center}
\Huge{Thanks}

\end{center}
\vspace{2em}
\begin{itemize}
\item \Large{\url{http://www.rathaxes.org/}}
\item \Large{\texttt{\#rathaxes} on IRC (\texttt{chat.freenode.net})}
\item \Large{\twitter}
\end{itemize}
\end{frame}

\begin{frame}{CNorm 3 / Codeworker}
CNorm 3.x is:
\begin{itemize}
\item A C Frontend for an EPITECH school's project;
\item Write in the codeworker scripting language.
\end{itemize}
Codeworker is:
\begin{itemize}
\item Allow to parse complex things (like ANTLR,Bison,...);
\item Easy for a one semester project;
\item Have some features that other tools don't have.
\end{itemize}
\end{frame}

\begin{frame}{CNorm 4 / Pyrser}
Pyrser is:
\begin{itemize}
\item A Python module
\item Have same features than codeworker
\item Release soon (need for CNorm 4)
\end{itemize}
CNorm 4.x is:
\begin{itemize}
\item The Rewrite of CNorm in python with Pyrser
\item Must be available for the next school session semester.
\end{itemize}
\end{frame}

\begin{frame}{Bibliography}
\rtxbibliography
\end{frame}

\end{document}
