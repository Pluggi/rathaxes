\documentclass{beamer}

\usepackage[francais]{babel}
\usepackage{rtxslides}

\title{\rtx}
\author{Louis Opter \\ \texttt{<louis@lse.epitech.eu>}}

\begin{document}

\begin{frame}
\titlepage
\end{frame}

\begin{frame}
\rmfamily{\itshape{«~\rtx\ est un \textcolor{rathaxesred}{langage dédié} qui
permet de \textcolor{rathaxesred}{décrire des pilotes} de périphériques. \rtx\
compile vers des modules noyaux écrits en C pour \textcolor{rathaxesred}{Linux},
\textcolor{rathaxesred}{Windows} et \textcolor{rathaxesred}{OpenBSD}.~»}}
\end{frame}

\begin{frame}{Développement de pilotes}
\begin{itemize}
\item code critique ;
\item double compétence électronique/programmation ;
\item spécifique à chaque système d'exploitation ;
\item nécessite un énorme de temps (d'auto) formation.
\end{itemize}
\end{frame}

\begin{frame}{Rathaxes}
\begin{itemize}
\item impose plus de vérification sur le code ;
\item sépare les compétences en électroniques et en programmation ;
\item unifie le développement pour toutes les plateformes.
\end{itemize}
\end{frame}

\begin{frame}{Architecture}
Drawings\ldots
\end{frame}

\end{document}
