\documentclass[a4paper,american]{rtxarticle}

\author{Louis Opter}

\title{AC `97 Sound Cards Implementations}

\rtxdoctype{State Of the Art}
\rtxdocversion{0.1}
\rtxdocstatus{Work In Progress}

\rtxdochistory{
0.1 & 11/23/2010 & Louis Opter & Skeleton and abstract \\
}

\begin{document}

\maketitle

\begin{abstract}
This document aims to explain how AC~`97 sound cards driver are implemented in
our different targets. AC~`97 stands for \emph{Audio Codec `97} and has been
developed by Intel in 1997. AC~`97 can be found in a wide range of sound cards
and motherboards chipsets. AC~`97 hardware is cheap and easy to find. In 2004,
this standard has been superseded by \emph{Intel High Definition Audio} (also
called \emph{HD Audio}, \emph{HDA-Intel} or \emph{Azalia}).
\end{abstract}

\section{Specification and existing drivers}

The AC~`97 specification tells us\cite[p. 11]{AC97spec} that AC~`97 is split in
several devices:
\begin{itemize}
\item a digital controller connected on the PCI bus of the computer;
\item an audio codec connected to the digital controller via an \emph{AC-link};
\item a modem codec connected to the digital controller via the AC-link.
\end{itemize}
The AC~`97 specification is, actually, intended for OEM manufacturers and covers
everything except the digital controller which makes it useless for \rtx.

A working AC~`97 driver can be found on each of our supported Operating System:
\begin{description}
\item[Linux:] \texttt{/sound/pci/intel8x0.c};
\item[OpenBSD:] \texttt{/sys/dev/pci/auich.c};
\item[Windows:] \ldots
\end{description}

\newpage

\rtxmaketitleblock

\rtxbibliography

\end{document}
