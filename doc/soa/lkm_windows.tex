\documentclass[a4paper,10pt]{article}
\usepackage[english]{babel}
\usepackage{listings}
\usepackage{color}
\definecolor{grey}{rgb}{0.95,0.95,0.95}
\usepackage{geometry}
\geometry{hmargin=2.5cm}

\title{Windows Loadable Kernel Modules}
\author{Mickael Dumont\\Tomas Suarez\\David Verriere}
\date{november 2007}

\begin{document}
\pagestyle{plain}
\lstloadlanguages{C}
\lstset{numbers=left, breaklines=true, basicstyle=\ttfamily,
numberstyle=\tiny\ttfamily, framexleftmargin=6mm, backgroundcolor=\color{grey},
xleftmargin=6mm, language=C, showspaces=false, showstringspaces=false}

\maketitle
\newpage
\tableofcontents
\newpage

\section{License}
This documentation is provided by the authors ``as is'' and any
express or implied warranties, including, but not limited to, the implied
warranties of merchantability and fitness for a particular purpose are
disclaimed. in no event shall the authors be liable for any
direct, indirect, incidental, special, exemplary, or consequential damages
(including, but not limited to, procurement of substitute goods or services;
loss of use, data, or profits; or business interruption) however caused and
on any theory of liability, whether in contract, strict liability, or tort
(including negligence or otherwise) arising in any way out of the use of this
software, even if advised of the possibility of such damage.

\section{Introduction}
This short article will introduce the way of handling
loadable kernel modules (LKM) on windows XP.

\section{Includes}
The windows ddk (driver develempement kit) provide the usefull include:
\begin{lstlisting}
#include <ntddk.h>
\end{lstlisting}

\section{Entry point/out point}
when the driver is loaded in the kernel the first function called is
{\it driver\_entry}. this function is in charge of creating everything needed
by the driver to function.\\
\begin{lstlisting}
NTSTATUS DriverEntry(IN PDRIVER_OBJECT DriverObject, IN PUNICODE_STRING RegistryPath)
\end{lstlisting}
The DriverObject is a data structure representing the driver. the accessible
members of this function
\begin{lstlisting}
PDEVICE_OBJECT DeviceObject
PDRIVER_EXTENSION DriverExtension
PUNICODE_STRING HardwareDatabase
PFAST_IO_DISPATCH FastIoDispatch
PDRIVER_INITIALIZE DriverInit
PDRIVER_STARTIO DriverStartIo
PDRIVER_UNLOAD DriverUnload
PDRIVER_DISPATCH MajorFunction[IRP_MJ_MAXIMUM_FUNCTION+1]
\end{lstlisting}
{\bf DeviceObject} is updated automatically by the {\it IoCreateDevice}\\
{\bf DriverExtension} a pointer to the extention structure DriverExtension>AddDevice is the only accessible member.\\
{\bf HardwareDatabase} is the path to the hardware configuration in the registry.\\
{\bf FastIoDispatch} is a pointer to the fastI/O structure whitch define the fastI/O entry point.\\
{\bf DriverInit} is the entry point for the DriverEntry.\\
{\bf DriverStartIo} is the entry point for the startI/O routine.\\
{\bf DriverUnload} the function called at unload time.\\
{\bf MajorFunction} is an arrray containing the function pointer to the dispatch ROUTINES (READ, WRITE,...).\\
In odrer to be able to unload the driver it is important to set the unload
field of the DriverObject.
\begin{lstlisting}
DriverObject-> DriverUnload = rathaxesUnload;
\end{lstlisting}

\section{Compile a module}
to compile a module under windows you need the ddk. You can find it on the
Microsoft website.
In order to build your driver two more files are needed: the sources file and
the makefile.\\
makefile:\\
\begin{lstlisting}
!INCLUDE $(NTMAKEENV)\makefile.def
\end{lstlisting}
sources:\\
\begin{lstlisting}
TARGETNAME=driverName
TAGETPATH=.
TARGETTYPE=DRIVER
INCLUDES= $(BASEDIR)\inc;.
SOURCES=driver.c
\end{lstlisting}
If you need multiple file as sources just write them whitesapce separated.

\section{Hello World}
\begin{lstlisting}
#include <ntddk.h>

void DriverUnload(PDRIVER_OBJECT pDriverObject)
{
        DbgPrint("Bye World\n");
}

NTSTATUS DriverEntry(PDRIVER_OBJECT DriverObject, PUNICODE_STRING RegistryPath)
{
        DbgPrint("Hello world\n");
	DriverObject->DriverUnload = DriverUnload;

        return(STATUS_SUCCESS);
}
\end{lstlisting}

To output message from the kernel we use the {\it DbgPrint()} function, hoever
this doesn't prouce any output on the command line. To view the output use a
debugger or the Dbgview program from www.sysinternals.com.

\subsection{Driver Registration}
Under Windows drivers are handled like normal services and have to be
registrated at the Service Control Manager. A driver need to be (and can)
registred only once in the Service Control Manager.\\
There are two way for registering a driver. You can write a loader using the
CreateService API or add the necessary registrykey by hand.
\subsubsection{CreateService API}
\subsubsection{Using regedit}
to register your driver you need to create a new key in your systeme and fill
4 registry key.\\
Firts create the new subkey: {\it
HKEY\_LOCAL\_MACHINE\textbackslashSystem\textbackslashCurrentControlSet\textbackslashServices\textbackslashdriverName}
then create the following key with the coresponding value:
{\bf DisplayName} driverName\\
{\bf ImagePath} c:\textbackslashpath\textbackslashto\textbackslashdriver\textbackslashdriverName.sys\\
{\bf Start} 3 (possible values are boot (0),  System (1), Auto (2), Demand
(3), Disabled (4))
{\bf Type} 1 (ie Kernel Driver)
\subsection{Load a driver}
The simplest way to load a driver is using a prompt to type the following
command:
\begin{lstlisting}
net start driverName
\end{lstlisting}
To unload the driver simply type:
\begin{lstlisting}
net stop driverName
\end{lstlisting}

\end{document}

