\documentclass[american]{rtxarticle}

\usepackage[utf8]{inputenc}

\title{User Documentation}
\author{Louis Opter}

\rtxdoctype{User Documentation}
\rtxdocstatus{Draft}
\rtxdocversion{0.1}

\rtxdochistory
{
0.1 & 04/07/2011 & Louis Opter & Initial release \\
}

\begin{document}

\maketitle

\begin{abstract}
\rtx\ is a Domain Specific Language to describes peripherals drivers. \rtx\
compiles to kernel modules written in C for Linux, Windows and OpenBSD.

This document explains how to setup the \rtx\ compiler on Windows and Unix like
platforms.
\end{abstract}

\rtxmaketitleblock

\tableofcontents

\section{Installation}

An installer exists for Windows on other platform only installations from the
sources are supported at the moment. Of course, you can also choose to install
\rtx\ from the sources on Windows if you intend to develop on \rtx\ itself from
Windows.

Although \rtx\ only generate drivers for Windows, Linux and OpenBSD it can be
installed and used from others Operating Systems including other BSDs and MacOS
X.

\subsection{Using the installer for Windows}

An installer is available for Windows. It includes everything you need to write
drivers using \rtx: the \rtx\ compiler and its documentation.

You can download the installer at: \url{http://rathaxes.googlecode.com/files/rathaxes-latest.exe}.

Simply execute it and follows the instructions on your screen.

To build and use the drivers generated by \rtx\ you will also need the
Microsoft Driver Development Kit.

\subsection{From a sources tarball on Unix}

If you are not on Windows you will have to install \rtx\ from a "source
release".

You can download a source tarball at: \url{http://rathaxes.googlecode.com/files/rathaxes-src-latest.tar.gz}.

You will also need to install CMake >= 2.6.

Extract the source tarball and cd into it, then you can install \rtx\ using:

\begin{verbatim}
$ mkdir build
$ cd build
$ cmake -DCMAKE_INSTALL_PREFIX=/usr/local/ -DCMAKE_BUILD_TYPE=RELEASE ..
$ sudo make install
\end{verbatim}

\emph{You will need to be root to issue "make install".}

\section{Generating your first driver}

TBD.

\section{Install the development version of \rtx}

You can install the latest version of \rtx\ if you need to have the latest bugs
and features or if you want to contribute to the project. This involve to
checkout the project from \texttt{rathaxes.googlecode.com} and to build it
manually.

\subsection{Pre-requisites}

To checkout and build the project you need to install the following softwares:
\begin{itemize}
\item Mercurial >= 1.5 (you can check the version with \texttt{hg --version}
      and use \href{http://tortoisehg.bitbucket.org/download/index.html}{TortoiseHg}
      on Windows);
\item Subversion (you need to use \href{http://www.sliksvn.com/en/download}{Slik
      SVN} on Windows which install);
\item CMake >= 2.6 (you can check the version with \texttt{cmake --version} and
      download it from \url{http://www.cmake.org/cmake/resources/software.html});
\item A compiler tool-chain (by installing the \texttt{build-essential} package on
      Debian-like GNU/Linux distribution ---for example---; on Windows you can
      install \href{http://www.microsoft.com/express/Downloads/#2010-Visual-CPP}{Visual Studio Express for C++}).
\end{itemize}

\subsection{Checkout the sources}

Open a shell (use Windows+R and type cmd.exe on Windows), and checkout the
project using:

\begin{verbatim}
$ hg clone https://rathaxes.googlecode.com/hg/ rathaxes
$ cd rathaxes
\end{verbatim}

You are now ready to build \rtx.

\subsection{Build \rtx}

On Unix like Operating Systems uses the following commands:

\begin{verbatim}
$ mkdir build
$ cd build
$ cmake ..
$ make
\end{verbatim}

On Windows one can do:

\begin{verbatim}
$ mkdir build
$ cd build
$ cmake -G "NMake Makefiles" ..
$ nmake
\end{verbatim}

\end{document}
