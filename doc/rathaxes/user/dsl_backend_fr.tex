\documentclass[french]{rtxreport}

\usepackage{color}
\usepackage{listings}

\author{David Pineau}

\title{Documentation du Langage : Backend}
\usepackage[utf8]{inputenc}
\rtxdoctype{Documentation}
\rtxdocref{backend\_documentation}
\rtxdocversion{0.1}
\rtxdocstatus{Draft}

\rtxdochistory{
0.1 & 07/04/2011 & David Pineau & Première version de la traduction depuis
                                  la version anglaise \\
}

\newcommand{\note}[1]{\marginpar{\scriptsize{\textdagger\ #1}}}



\definecolor{grey}{rgb}{0.90,0.90,0.90}
\definecolor{rBlue}{rgb}{0.0,0.24,0.96}
\definecolor{rRed}{rgb}{0.6,0.0,0.0}
\definecolor{rGreen}{rgb}{0.0,0.4,0.0}

\lstdefinelanguage{rathaxes}
{
    morekeywords={},
	sensitive=true,
	morecomment=[l][\color{rRed}]{//},
	morecomment=[s][\color{rRed}]{/*}{*/},
	morestring=[b][\color{rGreen}]",
	morestring=[b][\color{rGreen}]',
	keywordstyle={\color{rBlue}},
    commentstyle={\color{rRed}},
    moredirectives={import}
}[comments,strings,directives]

\lstdefinelanguage[front]{rathaxes}
{
}[keywords,comments,strings]

\lstdefinelanguage[middle]{rathaxes}
{
	morekeywords={interface, provided, required, optional, type, sequence,
                  variable},
    otherkeywords={::}
}[keywords,comments,strings]

\lstdefinelanguage[back]{rathaxes}
{
	morekeywords={with, template, type, sequence, decl, stmt, link, to, each},
	morecomment=[s][\color{rBlue}]{\$\{}{\}}
}[keywords,comments,strings]



\definecolor{lstbackground}{rgb}{0.95, 0.95, 0.95}
\definecolor{lstcomment}{rgb}{0, 0.12, 0.76}
\definecolor{lstkeyword}{rgb}{0.66, 0.13, 0.78}
\definecolor{lststring}{rgb}{0.67, 0.7, 0.13}
\definecolor{lstidentifier}{rgb}{0.1, 0.1, 0.1}

\lstset{
        language=rathaxes,
        tabsize=2,
        captionpos=b,
        emptylines=1,
        frame=single,
        breaklines=true,
        extendedchars=true,
        showstringspaces=false,
        showspaces=false,
        showtabs=false,
        basicstyle=\color{black}\small\ttfamily,
        numberstyle=\scriptsize\ttfamily,
        keywordstyle=\color{lstkeyword},
        commentstyle=\color{lstcomment},
        identifierstyle=\color{lstidentifier},
        stringstyle=\color{lststring},
        backgroundcolor=\color{lstbackground}
}

\lstset{alsolanguage={[back]rathaxes}}



\begin{document}

\maketitle

\rtxmaketitleblock

\tableofcontents

\abstract{
Rathaxes est un DSL (Domain Specific Language) ou langage dédié, ainsi qu'un
compilateur permettant de décrire un périphérique dans l'objectif de générer
le code C de son pilote, pour différents systèmes d'exploitation.

Développer un pilote implique d'écrire du code spécifique au périphérique en
plus d'écrire du code spécifique au système d'exploitation. La partie backend
du langage \rtx a pour objectif de faciliter l'écriture de code spécifique au
système d'exploitation.

En permettant la séparation de ces deux types de codes spécifiques, \rtx peut
générer de manière uniforme le code C du pilote pour chacun des systèmes
d'exploitation supportés par le backend.
}

\chapter{Le bloc “with”}

L'expression “with” décrit les conditions à respecter dans le but de
sélectionner le code spécifique écrit dans son bloc.

Ces conditions permettront par exemple d'expliciter au compilateur que le code
contenu dans le bloc “with” devra être utilisé seulement pour un système
d'exploitation spécifique (ex: Linux), avec une limitation de version
(ex: supérieure à 2.6.24).

Le code contenu dans le bloc de cette expression sera abordé plus tard dans ce
document.

\section{Les variables de configuration}

La partie intermédiaire du langage (middle-end), constituée d'interface, déclare
différentes variables. Nous appellerons ces variables les \emph{variables de
configuration}. Ces variables sont les éléments qui définissent quelles
implantations spécifiques à des systèmes d'exploitation (OS) peuvent être
utilisés lors de la génération ou non. En effet, puisque le contenu du bloc
“with” implante du code OS spécifique, il ne peut pas être utilisé pour
n'importe quel système d'exploitation. C'est pourquoi le processus de sélection
est articulé autour des ces \emph{variables de configuration}.

Le bloc “with” décrit pour quelles variables, et quelles valeurs de ces
variables, son contenu doit être sélectionné et éventuellement utilisé.

Lors de l'écriture d'une implantation spécifique à un système d'exploitation
pour \rtx, il est nécessaire de considérer la cible du code. Est-il spécifique
à un OS ? Spécifique à une version ? Quel sous-système ou bus implante-t-il ?
Chacune des réponses à ces questions est une information importante qui sera
contenue dans les \emph{variables de configuration} associées, et pourra
grandement influencer le processus de génération du pilote.

Maintenant, regardons comment écrire un bloc “with“. C'est en fait plutôt
simple :
\begin{lstlisting}
with os=Linux, version >= 2.6.24
{
    // Put some templates here.
}
\end{lstlisting}

Comme on peut le lire, le mot clef "with" doit être écrit en tout premier. Il
est alors suivi par une liste de \emph{variables de configuration} avec la
condition qu'elles doivent respecter. Chacune des conditions de cette liste est
séparée par une virgule. Ces conditions respectent trois règles :
\begin{itemize}
    \item Le nom de la variable de configuration doit être \emph{identique}
          au nom écrit dans l'interface que le bloc “with” implante, que ce
          soit partiellement ou dans son ensemble.
    \item La condition (ou comparaison) doit être une des 5 suivantes : 
        \begin{itemize}
            \item <: valide uniquement pour les nombres et les versions,
            \item <=: valide uniquement pour les nombres et les versions,
            \item =: valide pour les strings (chaînes de caractères),
                     les nombres et les versions,
            \item >=: valide uniquement pour les nombres et les versions,
            \item >: valide uniquement pour les nombres et les versions.
        \end{itemize}
    \item Les valeurs peuvent être de l'un de ces trois types :
        \begin{itemize}
            \item string: Écrites sous la forme d'un identifiant, sans être
                          encadrées de guillemets
                           (ex: Linux, Windows et non pas "Linux", "Windows"),
            \item nombre: un simple nombre (ex: 243, 7623, 1, 0),
            \item version: une suite de nombres séparés par des points
                           (ex: 2.6.24).
        \end{itemize}
\end{itemize}

Avec cela, nous sommes maintenant capables d'écrire une expression “with”
correcte qui permettra au compilateur de sélectionner notre code spécifique
pour une bonne génération de code C.

Il nous faut donc maintenant écrire le contenu de ce bloc “with” :
les templates (ou modèles de code).


\chapter{Définir un template}

Le contenu d'un template est le code qui fera partie (après résolution complète
du template) du code C généré. Le template décrit soit un type, soit une
séquence (l'équivalent \rtx de fonction), et quelle sera leur implémentation
réelle en C. Nous n'allons pas décrire le code C instrumenté contenu par le
template (donc dans son bloc de code), puisque celui-ci fait l'objet d'un autre
chapitre. Sachez cependant que le bloc du mot clef template accueille du code
C instrumenté pour les besoins du langage \rtx.


\section{Le prototype d'un template}

Pour écrire un template, il faut commencer par le mot clef "template". Il est
suivi d'un second mot clef, identifiant le type de template (on différencie un
template implantant un type d'un template implantant une séquence par ce second
mot clef). Ensuite vient ce que nous appellerons le \emph{prototype de
template}. Ce prototype inclue le nom de la séquence ou du type (ou nom du
template), qui sera l'identifiant utilisé par le langage rathaxes pour le
référencer, ainsi que des types et noms de paramètres, définissant les
\emph{paramètres du template}.

Le \emph{prototype de template} est donc la partie qui identifie l'élément
implanté par le template. Faites cependant attention à ne pas confondre un
template et son implantation : le template ne se transcrit pas nécessairement
en fonction C, même s'il décrit une séquence.

Le nom du template est un simple identifiant alphanumérique (comprenant aussi
l'underscore). Les paramètres du template sont en réalité des variables dont
les types proviennent du langage \rtx . Ces variables sont utilisables pour
permettre, par interaction entre templates, de générer du code C valide et
cohérent.

Voici comment on peut écrire un template de séquence :
\begin{lstlisting}
with os=Linux, version >= 2.6.24
{
    // here we can identify the bnf structure: 
//  pkeyword skeyword name(ParamType1 param1, ParamType2 param2)
    template sequence foo(Context ctx, register reg)
    {
        // put the template's code here'
    }
}
\end{lstlisting}

\section{Surcharge de template}

\rtx supporte la surcharge de template, signifiant qu'une séquence ou un type
(identifié grâce au nom du template) peut être écrit pour différents types de
paramètres. On pourrait donc écrire deux templates nommés "foo" qui recevraient
des types de paramètres différents :

\begin{lstlisting}
with os=Linux, version >= 2.6.24
{
    // this one here manipulates a register
    template sequence foo(Context ctx, register reg)
    {
        // put the template's code here'
    }

    // This one only uses context's informations
    template sequence foo(Context ctx)
    {
        // put the template's code here'
    }
}
\end{lstlisting}


\chapter{Variables templates}

Nous pouvons donc écrire des templates. Super. Seulement, nous aimerions
utiliser ces templates comme de vrais modèles de code, qui pourra engendrer la
génération de codes différents selon les situations. C'est en effet le but des
\emph{variables templates}. Elles peuvent être manipulées de trois manières
différentes, encadrées par les balises "\$\{" et "\}". Penchons nous sur ces
trois utilisations possibles.


\section{Structure}

Pour commencer, chaque \emph{variable template} est structurée d'une manière
spécifique. Par exemple, le type register contient un nom, une adresse, et
éventuellement une collection de champs. La structure de chaque type template
est décrite dans le template de type associé. Il faut donc se référer au
template en question afin d'obtenir plus d'informations au sujet de la
structure de cette \emph{variable template}.


\section{Concaténation d'identifiant}

La première et plus simple façon d'utiliser une \emph{variable template} est de
l'utiliser afin de générer des identifiants C contenant le nom associé à la
variable, ou n'importe lequel de ses champs directement concaténé dans le code
C généré.

Prenons l'exemple du type buffer. Nous pourrions utiliser plusieurs buffers de
types différents dans notre code de périphérique. Si nous utilisions alors une
fonction générique, il serait possible d'obtenir un comportement indéfini et
souvent indésirable sur au moins l'un d'entre eux. On pourrait donc utiliser
une propriété du buffer manipulé pour générer des fonctions adaptées à chacun.

C'est d'ailleurs la raison pour laquelle la syntaxe accepte des préfixes et des
suffixes aux balises "\$\{"\ldots"\}". Ainsi, écrire le code suivant avec un
registre nommé eeprm :
\begin{lstlisting}
with os=Linux, version >= 2.6.24
{
    // this one here manipulates a register
    template sequence foo(Context ctx, register reg)
    {
        int     the_${reg.name}_flag;
    }
}
\end{lstlisting}

pourrait alors générer (pour les bonnes valeurs de \emph{variables de
configuration}) :
\lstset{language=C}
\begin{lstlisting}
int the_eeprm_flag;
\end{lstlisting}
\lstset{language={[back]rathaxes}}

Cette technique peut bien évidemment être utilisée pour tout type de
concaténation : identifiants, valeurs écrites en dur dans le code généré, etc...

\section{Liaison de templates}

Il arrive parfois que l'on ait besoin de faire appel à d'autres templates afin
par exemple de générer un algorithme complet et cohérent. Cela peut notamment
survenir dans le cas d'un découpage logique fin des différentes unités
algorithmiques constituant le template. Nous avons donc besoin d'un moyen
d'expliciter le lien vers un autre template. C'est une sorte d'appel entre
templates.

Pour ce faire, voici la syntaxe générale :
\begin{lstlisting}
${link template_variable_list to template_prototype};
\end{lstlisting}

La liste de variables templates, séparée par des virgules, doit être constituée
de variables templates accessibles au sein du template actuel. Le prototype de
template, lui, décrit quel template lier au notre. Attention, les variables
templates et leur ordre doit correspondre aux types (et leur ordre) attendus
par le template lié.

L'effet de cette syntaxe est d'inclure le code généré du template lié au sein
du code généré du template liant.

On pourrait donc écrire :
\begin{lstlisting}
with os=Linux, version >= 2.6.24
{
    template sequence foo(Context ctx, register reg)
    {
        ${link reg to bar(register)};
    }

    template sequence bar(register reg)
    {
        int     the_${reg.name}_flag;
    }
}
\end{lstlisting}


%% TODO FIXME
%% The rest of the "EACH" syntax is to be defined and then
%% the documentation written....
TODO : syntaxe du EACH
%%The “each” syntax is based on the same model as the “link” syntax, but allows to
%%link each element of a collection to the same template. This prevents writing
%%loops that would be error-prone. It would be written like this:

%%\begin{lstlisting}
%%${each template_variable as alias to template_prototype};
%%\end{lstlisting}


\subsection{Transmission de la configuration}

Maintenant on sait manipuler les \emph{variables templates}, afin de générer du
C à partir du C instrumenté de \rtx. Il y a cependant une limitation évidente :
les \emph{variables de configuration}. En effet, puisqu'un template est
sélectionné en fonction de leurs valeurs, le template lié doit être implanté
pour les variables et valeurs qui correspondent au template liant.


\chapter{Variables templates accessibles}


\section{Limitations}

On sait que l'on peut utiliser les paramètres du template en tant que variables
templates. Malheureusement, il arrive parfois qu'on ai besoin de plus : par
exemple, d'utiliser une variable déclarée en C correspondant à l'implantation 
d'un type \rtx. C'est pourquoi le langage offre des variables templates
contextuelles.


\section{Variables templates contextuelles}

Les variables templates contextuelles peuvent être séparées en deux catégories :
\begin{enumerate}
    \item Variables globales,
    \item Variables locales.
\end{enumerate}

La variable "global" (pour les variables globales) contient toutes les variables
de configuration. Chacun doit être accédée en tant que champ de la variable 
template "global", le nom du champ étant le nom de la variable de configuration.

La variable "local" (pour les variables locales) contient toutes les variables
déclarées dans le code C du template. Ceci permet notamment l'utilisation
d'algorithmes \rtx pour ces variables. De la même manière que pour les
variables globales, on y accède au travers de la variable "local", avec leurs
nom en tant que champ de cette dernière.

Par exemple, afin de transmettre une variable C à un template lié, on écrit :
\begin{lstlisting}
with os=Linux, version >= 2.6.24
{
    template sequence multiset(register reg)
    {
        {
            // declare a tmp variable of the same type as "reg"
            ${register}     tmp;
            // Here we use "local.tmp" meaning we use the C variable tmp
            tmp = ${link reg to get(register)};
            tmp |= (val & mask);
            ${link reg, local.tmp to set(register, register)};
        }
    }

    template sequence get(register reg)
    {
        // read a byte
        inb(${reg.addr})
    }

    template sequence set(register rreg, register lreg)
    {
        // Set the real register to the temporary one's value.
        outb(${rreg.addr}, ${lreg.name})
    }
}
\end{lstlisting}

Ces fonctionnalités impliquent bien évidemment qu'il est interdit de nommer une
variable de template "global" ou "local".


\chapter{Annexe : BNF}

Afin de résumer la syntaxe particulière de la partie backend du langage \rtx,
voici la BNF associée :

\lstset{}
\begin{lstlisting}
// This is the root entry point of the BNF

with_block ::= "with" config_value_list '{' [ template_block ]* '}'

config_value_list ::= [ config_value ]*

config_value ::= config_variable config_condition config_value

config_variable ::= identifier

config_condition ::= '<' | "<=" | '=' | "=>" | '>'

config_value ::= identifier | version

template_block ::= "template" type_key template_prototype
                    '{' [ C | template_placeholder ]* '}'

C ::= // this is the whole C syntax that we wont define here...

type_key ::= "type" | "sequence"

template_prototype ::= identifier '(' [tpl_var_list]? ')'

tpl_var_list ::= tpl_var_type tpl_var_id [ ',' tpl_var_type tpl_var_id ]*

template_placeholder ::= [ prefix ]? [ template_braced_code ]+ [ postfix ]?

template_braced_code ::= "${" tpl_each | tpl_link | tpl_var '}'

tpl_each ::= "each" tpl_var "as" alias "in" tpl_proto

tpl_link ::= "link" tpl_var_list "to" tpl_proto

tpl_var ::= identifier [ '.' tpl_var_field ]*

tpl_var_field ::= identifier

tpl_proto ::= identifier '(' [tpl_proto_var_list]? ')'

tpl_proto_var_list ::= tpl_var_type [ ',' tpl_var_type ]*

tpl_var_type ::= identifier

tpl_var_id ::= identifier

version ::= number [ '.' number ]*

number ::= [ '0'..'9' ]*

identifier ::= [ 'a'..'z' | 'A'..'Z' | '_' ]
               [ '0'..'9' | 'a'..'z' | 'A'..'Z' | '_' ]*

\end{lstlisting}





\end{document}
